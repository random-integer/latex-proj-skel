\section{section two}

\begin{tcolorbox}[colback=myteal!5!white,colframe=myteal!75!black,title=\textsf{課題2}]
	出力電圧を
	\begin{equation*}
		e(t) =
		\begin{cases}
			\ 0,   & \text{if}\ t < 0   \\
			\ e_0, & \text{if}\ t \ge 0
		\end{cases}
	\end{equation*}
	のようにするには,どう設定すればよいか.
\end{tcolorbox}

\begin{lstlisting}[language=C++]
#include <bits/stdc++.h>
using namespace std;

int main() {
	cout << "Hello world!" << endl;
	// 色は匂へど 散りぬるを
	cout << "我が世誰ぞ 常ならむ" << endl;
	// 有為の奥山 今日越えて
	// 浅き夢見し 酔ひもせず
	return 0
}
\end{lstlisting}

Lorem ipsum dolor sit amet.

This is an old Japanese poem called \emph{Iroha}: \\
いろはにほへと\ ちりぬるを \\
わかよたれそ\ つねならむ \\
うゐのおくやま\ けふこえて \\
あさきゆめみし\ ゑひもせす

連立不等式を解くぜ!
\begin{equation}
	ax^2 + bx + c = 0
\end{equation}
水とは、化学式 H₂O で表される、水素と酸素の化合物である。日本語においては特に湯と対比して用いられ、液体ではあるが温度が低く、かつ凝固して氷にはなっていないものを言う。また、液状の物全般を指す。なお、湯は温かい水を指していう。
\begin{equation}
	x = \frac{-b \pm \sqrt{b^2 - 4ac}}{2a}
\end{equation}
だよ。

Last night of all, \\
When yond same star that's westward from the pole \\
Had made his course t' illume that part of heaven \\
Where now it burns, Marcellus and myself, \\
The bell then beating one—
