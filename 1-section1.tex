\section{我不知道}

六經之道同歸,而禮樂之用為急,治身者斯須忘禮,則暴嫚入之矣。為國者一朝失禮,則荒亂及之矣。人函天地陰陽之氣,有喜怒哀樂之情,天禀其性,而不能節也。聖人能為之節而不能絕也。故象天地而制禮樂,所以通神明,立人倫,正情性,節萬事者也。

RLC回路は、抵抗(R)、コイル(L)、コンデンサ(C)を直列または並列に接続した回路のことです。

The quick brown fox jumps over the lazy dog.

データの符号化
\begin{enumerate}
	\item (漢字モードの場合)まずQRコードに入れる文字列データをシフトJISのバイト列に変換し,続いてQRコード用のビット列(13ビット長)に変換する.
	\item ヘッダ情報(モード指示子・文字数データ)に続いて,13桁のビット列を並べる.
	\item 並べ終わったら,最後に「終端パターン」と「埋め草ビット」を並べる.
	\item シンボルのデータコード語の容量いっぱいになるまで,「埋め草コード語」を埋めていく.
	\item 並べたビット列を8ビット(=1バイト)ずつに区切って,符号化は完了.
\end{enumerate}

citation \cite{sizesuu}
