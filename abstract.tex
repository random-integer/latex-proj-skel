\begin{abstract}

	孟子告齊宣王曰:「君之視臣如手足,則臣視君如腹心;君之視臣如犬馬,則臣視君如國人;君之視臣如土芥,則臣視君如寇讎。」

	王曰:「禮,為舊君有服,何如斯可為服矣?」

	曰:「諫行言聽,膏澤下於民;有故而去,則君使人導之出疆,又先於其所往;去三年不反,然後收其田里。此之謂三有禮焉。如此,則為之服矣。今也為臣。諫則不行,言則不聽;膏澤不下於民;有故而去,則君搏執之,又極之於其所往;去之日,遂收其田里。此之謂寇讎。寇讎何服之有?」

	吾輩は貓である。名前はまだ無い。

	どこで生れたかとんと見當がつかぬ。何でも薄暗いじめじめした所でニャーニャー泣いていた事だけは記憶している。吾輩はここで始めて人間というものを見た。しかもあとで聞くとそれは書生という人間中で一番獰惡な種族であったそうだ。


	人類社会のすべての構成員の固有の尊厳と平等で譲ることのできない権利とを承認することは,世界における自由,正義及び平和の基礎であるので,

	人権の無視及び軽侮が,人類の良心を踏みにじった野蛮行為をもたらし,言論及び信仰の自由が受けられ,恐怖及び欠乏のない世界の到来が,一般の人々の最高の願望として宣言されたので,

	人間が専制と圧迫とに対する最後の手段として反逆に訴えることがないようにするためには,法の支配によって人権を保護することが肝要であるので,

	諸国間の友好関係の発展を促進することが肝要であるので,国際連合の諸国民は,国連憲章において,基本的人権,人間の尊厳及び価値並びに男女の同権についての信念を再確認し,かつ,一層大きな自由のうちで社会的進歩と生活水準の向上とを促進することを決意したので,

	加盟国は,国際連合と協力して,人権及び基本的自由の普遍的な尊重及び遵守の促進を達成することを誓約したので,

	これらの権利及び自由に対する共通の理解は,この誓約を完全にするためにもっとも重要であるので,

	よって,ここに,国連総会は,

	社会の各個人及び各機関が,この世界人権宣言を常に念頭に置きながら,加盟国自身の人民の間にも,また,加盟国の管轄下にある地域の人民の間にも,これらの権利と自由との尊重を指導及び教育によって促進すること並びにそれらの普遍的措置によって確保することに努力するように,すべての人民とすべての国とが達成すべき共通の基準として,この人権宣言を公布する.

\end{abstract}

