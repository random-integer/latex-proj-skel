\begin{abstract}

	孟子告齊宣王曰:「君之視臣如手足,則臣視君如腹心;君之視臣如犬馬,則臣視君如國人;君之視臣如土芥,則臣視君如寇讎。」

	王曰:「禮,為舊君有服,何如斯可為服矣?」

	曰:「諫行言聽,膏澤下於民;有故而去,則君使人導之出疆,又先於其所往;去三年不反,然後收其田里。此之謂三有禮焉。如此,則為之服矣。今也為臣。諫則不行,言則不聽;膏澤不下於民;有故而去,則君搏執之,又極之於其所往;去之日,遂收其田里。此之謂寇讎。寇讎何服之有?」

	吾輩は貓である。名前はまだ無い。

	どこで生れたかとんと見當がつかぬ。何でも薄暗いじめじめした所でニャーニャー泣いていた事だけは記憶している。吾輩はここで始めて人間というものを見た。しかもあとで聞くとそれは書生という人間中で一番獰惡な種族であったそうだ。

\end{abstract}

